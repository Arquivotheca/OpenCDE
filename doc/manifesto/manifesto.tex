\input eplain

%%%%%%%%%%%%%%%%%%%%%%%%%%%%%%%%%%%%%%%%%%%%%%%%%%%%%%%%%%%%%%%%%%%%%%%%
%% Note on citation and formatting
%
% The form of this document is based on the Note + Extended Bibliography
% style of the Chicago Manual of Style. The formatting of the general
% document aims to be a combination of old-style book looks and
% formatting with the addition of a hailback to the BSD published
% documentation, which was essentially typeset mandoc pages.
%
%%%%%%%%%%%%%%%%%%%%%%%%%%%%%%%%%%%%%%%%%%%%%%%%%%%%%%%%%%%%%%%%%%%%%%%%

% Uncomment these fonts if you want something more stylish
\font\rm = "Baskerville-Nova" at 12pt
\font\bf = "Baskerville-Nova/B" at 12pt
\font\it = "Baskerville-Nova/I" at 12pt
\font\tt = "APL385 Unicode" at 12pt
\font\smallrm = "Baskerville-Nova" at 10pt
\font\smallit = "Baskerville-Nova/I" at 10pt
\font\smallbf = "Baskerville-Nova/B" at 10pt
\font\smalltt = "APL385 Unicode" at 10pt
\font\hdgfnt = "Baskerville-Nova/B" at 14pt
\font\ttlfnt = "Baskerville-Nova" at 17pt

% If those don't work for you, every X system should have these
%\font\rm = "Century Schoolbook L" at 12pt
%\font\bf = "Century Schoolbook L/B" at 12pt
%\font\it = "Century Schoolbook L/I" at 12pt
%\font\tt = "Luxi Mono" at 12pt
%\font\smallrm = "Century Schoolbook L" at 10pt
%\font\smallit = "Century Schoolbook L/I" at 10pt
%\font\smallbf = "Century Schoolbook L/B" at 10pt
%\font\smalltt = "Luxi Mono" at 10pt
%\font\hdgfnt = "Century Schoolbook L/B" at 14pt
%\font\ttlfnt = "Century Schoolbook L" at 17pt

\def\heading#1{%
  \bigskip\bigskip
  {\leftskip = 0in
    \def\curheadtxt {#1}
    \noindent{\hdgfnt #1}
    \par}
  \bigskip}

\def\hpref#1{%
  \noindent{\bf #1.\quad}}
\def\pref#1{%
  \bigskip\hpref{#1}}

\leftskip = 1in
\def\curheadtxt{}
\everyfootnote = {\smallrm }

\let\footnote = \numberedfootnote

\headline{%
  \ifnum \pageno > 1
    \ifodd\pageno
      {\smallit \curheadtxt} \hfill {\smallrm \the\pageno}\par
    \else
      {\smallrm \the\pageno} \hfill {\smallit OpenCDE Manifesto}\par
    \fi
  \else
    \hfil
  \fi}
\footline{\hfil}

\rm

\null
\vskip 2in
\hrule
\medskip
\centerline{\ttlfnt OpenCDE Manifesto: Guiding Principles and Project Goals}
\medskip
\hrule
\vskip 1in
{\flushright
{\hdgfnt Contributors}

Aaron W. Hsu
{\tt arcfide@sacrideo}

Karsten Pedersen
{\tt kpedersen@opencde.org}}

\vfill

\noindent
{\leftskip = 0in
Copyright (c) 2011 The OpenCDE Project. All rights reserved.\par}
\break

\heading{Purpose of this Document}

OpenCDE has the potential to scale and grow as a large and
comprehensive project. However, many projects loose sight and focus as
they gain contributors and different bodies push with their own agendas.
This document aims to provide a cohesive vision of the direction and
intentions of OpenCDE developers with respect to where and how OpenCDE
should move forward. In terms of scope, we intend to omit nothing of
import, and this document has no requirements of being short. However, it
ought to remain self consistent and broad enough to enable developer
freedom where important without losing the focus of the project.

\heading{A History of CDE}

Before the advent of CDE, UNIX vendors had no consistent desktop
standard by which they could develop and share applications. There were
many different toolkits and styles of programming, and no consistency
among desktop interfaces. This lead to numerous problems. The OpenGroup
describes CDE in the following terms:

\medskip
{\narrower\it
The Common Desktop Environment (CDE) is an integrated graphical user
interface for open systems desktop computing. It delivers a single,
standard graphical interface for the management of data and files (the
graphical desktop) and applications. CDE’s primary benefits—deriving
from ease-of-use, consistency, configurability, portability, distributed
design, and protection of investment in today’s applications—make open
systems desktop computers as easy to use as PCs, but with the added
power of local and network resources available at the click of a mouse.%
\numberedfootnote{OpenGroup, {\smallit CDE}, http://www.opengroup.org/cde/.}
\par}
\medskip

\noindent
Indeed, CDE was the standard among desktop environments in UNIX 
Workstations for some time, especially during the peak of commercial
UNIX development. CDE had a reputation as the standard desktop, though
it did not necessarily have a reputation as the best, most efficient,
fastest, or the like. And especially, in the era of thousand dollar
per seat licensing costs, it was not cheap. It was, in a sense, open. The
standards on which it were based were open and unencumbered, and the
basic protocols that it used were also fairly standard. However, in
terms of the code itself, it was highly restricted. Gaining access to the
source code was expensive, and it was even more expensive to obtain
distribution rights.

Since many companies made their living on CDE contracts, this did not
change even though the world changed around CDE. Indeed, the OpenGroup
was able to open source Motif, the core toolkit upon which CDE is based,
but it was not able to open source CDE. A petition is still outstanding
to accomplish this, but the current expectation is the need for a large
one time payment to buy the sources. To many, this is too little, too
late. The world has moved on, in many ways, and users’ expectations have
also changed over the years, but CDE development has remained largely
stagnant. It should be noted, however, that CDE was never at the cutting
edge of desktop features. The point of CDE was stability,
predictability, and consistency. Leveraging investment in existing
knowledge and training was important. Leveraging existing application
investments also played into the design decisions surrounding CDE.

In the modern era, CDE has largely grown irrelevant because it simply
costs too much. Desktop environments like XFCE may have started as CDE
lookalikes, but they have changed so dramatically that they no longer
represent a close approximation to CDE, and they do not share the same
goals as the original CDE project. Gnome and KDE were never designed
with the goals of CDE in mind, and some would argue that this is a good
thing. 

Nonetheless, the OpenCDE project was started by Karsten Pedersen in an
effort to recreate the glory of CDE, its niceties, without the
associated drawbacks. 

\heading{The Principles}
The OpenCDE design principles are summarized below, roughly
in order of descending importance. Note that while these
principles summarize the situation, they do not represent ironclad
commandments etched into stone, never to be altered, deviated from, or
disregarded entirely.

\numberedlist

\li Provide a modern integrated graphical desktop experience providing
modern equivalents to traditional CDE end-user benefits including the
following:

  \unorderedlist
  \li Standard graphical interface with consistent look and feel
  \li Single graphical interface for the management of data and files
  \li Single graphical interface for the management of applications
  \li Easy-to-use%
      \numberedfootnote{This does not necessarily mean intuitive.}
  \li Consistency
  \li Configurability
  \li Portability
  \li Distributed Design
  \li Protection of Investment (A.K.A. Stability, Compatibility)
  \li Easy access to powerful resources
  \endunorderedlist

\li Provide high-performance, low-latency, low footprint computing
without the associated loss in functionality.

\li Integrate with standards and best practices, rather than reinventing
the wheel.

\li Integrate rather than segregate the desktop from the rest of UNIX.

\li Provide best-anywhere documentation, both at an user and at a
developer’s level. 

\li Appeal to developers and power users 

\li Emphasize quality, good design, and stability over rapid progress,
feature creep, and desktop fashion.

\li Favor function and user efficiency over eye-candy

\endnumberedlist

\heading{Clarifying Traditional CDE End-user Benefits}

\hpref{Standard Graphical Interface}
At the time CDE was being designed, UNIX had many different
graphical existed. Applications built on one toolkit would behave 
entirely differently to others, and there were little standards among
applications, either command line or graphical. One might be written in
straight Xlib, another in OpenLook, and another in something else
entirely. Command line options were not standard or consistent, and the
style of each application’s visual interface usually differed from
another application significantly. Just look at some common X
applications to see the situation even among applications written for a
single Xlib toolchain. Xman(1), Xfontsel(1), Xmore(1), or Xfig(1). All
of these have slightly different behaviors and it’s not clear or
intuitive what actions will do what on what systems.

CDE provided a solution to this by standardizing the look and feel of
applications inside of a single standard: Motif and CDE style guides.
The style guides of these two systems (CDE was built using Motif)
converged in the 2.1 version of CDE, further improving CDE’s ability to
offer a consistent look and feel. It achieved this with the help of a
layering of technologies. The Xlib libraries provided the basic
connections to the X server, but on top of that, the Xt toolkit provided
the basic infrastructure for object-oriented widget design, enabling
sub-classing and attribute sharing among widgets. The Motif toolkit took
advantage of Xt to build a standard set of widgets that emulated the
interface styles and features that people expected coming from Personal
computers. 

Beyond just a look and feel, though, CDE provides access to common
elements that the user would likely want to use. It enabled
administrator’s to more easily control the general look and feel of
workstations across a computing grid, and enabled a consistent theme
across international boundaries with internationalization.

\pref{Single interface for the management of data and files}
CDE provided a common interface, both programmatic and user-level for
managing files and data as it crossed application and network
boundaries. At its simplest, it provided a graphical interface for the
management of an user’s file system, through a simple file manager.
However, this file manager also spoke a communication protocol that
enabled users to graphically send data from files and directories to
other applications with a simple mouse action, rather than requiring
complicated shell pipes and the like. Applications that spoke the data
communication language of CDE could accept drag and drop elements from
the file manager, and vice versa. 

CDE’s network services enabled this same easy transparency across
applications and computers, so that server client code could
communicate more easily, and file sharing became something that the user
could accomplish with the mouse and pointer, rather than requiring the
editing of complex programs and daemons under the hood. 

This consistent programming interface meant that user’s on the UNIX
desktop now had access to drag and drop functionality that was akin to
or more powerful than what was available on systems like the Apple
Macintosh. Using a single interface and common, consistent desktop
motions that an user could remember, data could easily be transmitted
from one application to another through things like cut and paste, drag
and drop, and automatic interprocess communication.

\pref{Management of applications}
In the same way that CDE enabled consistent, predicatable data
interchange between applications, it also enabled the user to have a
consistent way of launching, accessing, and managing applications, both
in terms of look as well as in functionality.

The most obvious element to this was the inclusion of the front panel.
This was a configurable launch pad that enabled the user to set up
different application actions, launch the application of their choice,
and even embed applications into the panel itself.

Further adding to this, CDE introduced the ability to create custom
actions for applications, which was basically a way of starting up an
application the way you wanted it, with the appropriate icons and other
behaviors all in place. These actions where encapsulated into icons or
objects in the file system that were displayed in the desktop or in the
front panel. This prevented the user from needing to deal with things
like aliases and internal shell scripts for simple things like launching
graphical applications.

For system administrator's, it made it easier to install and manage
applications at a multi-system level, so that applications would be
available on multiple workstations over the network, and that they would
have the same look and feel across those workstations.

CDE's integration technologies made it possible to easily integrate
existing software into the CDE desktop, so that it would play nicely
with the user's expectations, even if the application itself was not
necessarily written with CDE in mind. OpenLOOK applications come to mind
here.

CDE also provided session management, and a protocol to enable
applications to save their own sessions (through standard X protocols)
to preserve state between logins. The user would then be presented with
the same desktop he logged out with in their previous session. More
complex forms of session management were also possible. 

As funny as it may sound today, the provision of an icon editor also
made managing applications easier, because the user could create or
import images and icons that they wanted to use for a given application,
especially considering that some applications may not have had an icon
associated with them in the first place. Today, this might not be
considered such a special idea, because of the pervasive nature of icons
and graphical applications, but this was something to be considered back
then.

\pref{Easy to Use}
CDE hailed itself as being easy to use. That is not to say that it was
necessarily intuitive. Indeed, it could be said that any desktop
environment is unintuitive to the user who is used to another. On the
other hand, CDE did design itself so that user's from Personal Computers
could easily transition their basic knowledge of computer operations
onto the more powerful UNIX system without the learning curve usually
associated with other desktop shells, such as the Korn Shell and the
command line. The ease of use came in the form of a consistent interface
that could be easily taught to people, and which worked reliably across
systems. The graphical nature of the system, with its graphical
configuration utilities, and other user-friendly aids, made it possible
to operate a UNIX workstation in a more comfortable and efficient way.

System administrators also found the benefits to this in terms of easier
application management and easier deployment of resources. It reduced
training necessary to get users productive on the system, and enabled
user's to move from one UNIX operating system to another without having
to worry about the underlying quirks fo the new operating system,
because CDE provided a common interface for them all. 

Beyond just how easy the system was to use, the CDE system provided a
single, unified documentation system that was linked into all the
applications, written in a single standard language, and enabled users
to browse and search for various answers inside of a single information
help system. The system provided a one stop shop for help in any of the
desktop applications, and its hypertext nature made it easier to
maneuvour through.

\pref{Consistent}
One aim of CDE was to provide consistency. Widgets looked the same and
worked the same across multiple operating systems. Back then, an
important element was consistency across platform on the UNIX systems,
and not just consistency within the desktop itself. The user was
expected to be able to learn how CDE worked, and then work effectively
on any number of Operating systems without having to be retrained.
It also provided a level of consistency on the desktop in the form of
the object oriented emphasis of the system. Icons and mouse based
actions were familiar, and applications used the same sort of actions
across the desktop. A drag and drop was the same in one application as
it was in another, and you could configure and change the look and feel
of each application using the same Xresources based settings, which
could also be set graphically, much in the form of a registry editor of
Windows. Other options besides look and feel could also be set through
these resources, giving a common interface for managing all sorts of
setttings in applications, which could be layered in the form of
defaults, system level settings, and user-level overrides. 

Further more, documentation was all done using a single standard
language and was interfaced through a single standard programming
interface. Applications would be able to install their manuals and
guides and then the user could browse it with their help system. 

Another feature added into the 2.1 version was a common printing
interface, that allowed any application to communicate a rendered version
of a widget and have it sent to the printer instead of the screen.

\pref{Configurability}
CDE was very configurable. Whether through configuration files or
through the graphical configuration managers, user's could control many
of the aspects of their system. It allowed for all sorts of flexibility
that users of UNIX systems have come to expect. These configurations
meant that whatever the situation, CDE could probably be fit within the
existing needs of your organization. It helped that Motif and X itself
was extremely flexible. CDE made it possible for users and
administrators to access this flexibility in a relatively easy manner
compared to the traditional methods.

\pref{Portability}
CDE was built in collaboration with a number of corporations, each with
their own UNIX operating system. Before CDE and its standardization,
users would need to adjust to each operating system and learn its quirks
before becoming productive. CDE provided a single, easy to use, and
graphical interface to all of the Operating systems, so that user's
could become productive with their workstations regardless of the
underlying operating system. If it could run X, it could run CDE, and
you could have the benefits of the same system across many different
platforms.

\pref{Distributed Design}
CDE was designed in an era that was seeing the change from time-shared
mainframe systems into distributed client-server systems where the
graphical rendering occurred on either lightweight or heavyweight
graphical display servers that hosted the display of an application that
was hosted on an application server. CDE's design enabled it to ease the
lives of user's who wanted to work in a heavily networked, client-server
world. Resources available on other machines, such as applications or
data could be accessed easily with CDE, even though the other servers
may not have been running CDE themselves. 

The system of CDE itself was to some extent, distributed. This enabled
the pieces of the system to be addressed at the appropriate level by the
appropriate person, rather than requiring a single monolithic approach.
This can be seen in the form of how layered the configuration is, so
that an user can use their own settings to override system ones, and
system administraters can make site-wide changes without having to worry
about default settings of the application or user-level changes. 

This designed enabled the whole CDE system to appear as a unified system
to the user, when in reality, many elements might have been distributed
across machines all over. 

\pref{Protection of Investment}
When CDE came along, companies had already invested huge sums of money
into existing applications. If their existing applications did not work
or integrate nicely into the CDE environment, CDE would have been
worthless. CDE enabled administrators and users to effectively integrate
current applications that they had written easily. This meant that an
OpenLOOK application could be wrapped and integrated in look and feel
with the rest of CDE.

This protected the investments of companies in existing custom
applications and various other workflows; CDE would integrate and merge
with existing needs rather than requiring an all or nothing commitment
from the company.

\pref{Easy access to powerful resources}
As mentioned before, CDE was built on UNIX, and this enabled it to
access the power of UNIX's networked nature. Something that PC users
couldn't yet dream of. This networked capability was sometimes hard to
get to, though, and CDE contributed to UNIX by providing a consistent
and mouse-oriented means of accessing these powerful networked and local
services.

\heading{What does CDE do right?}

The most important thing CDE did right, historically, is the standard,
high-quality level of integration that the system could provide. That
is, it provided consistency and regularity to a degree not found in
other options. Its support of networking was, of course, a plus.
Moreover, as time went on, CDE remained stable, reliable, and did not
break compatibility. This was vitally important for companies where
sources were usually not available, and systems had to remain ABI as
well as API compatible or they would not run. This ABI stability is
something that gave CDE a strong endurance. 

Ironically, in today's world, CDE's ABI stability is now one of its best
features. Other desktops have supplanted CDE as the “standard” among
desktop systems, and GTK might arguably have that role now, but Motif is
still the standard for mission-critical applications, and CDE is still
the only full scale desktop environment targeting this stable, reliable
toolkit. 

CDE can now also claim to be one of the lightest weight full-featured
desktop systems, in large part because it has avoided the feature creep
of other desktop environments.%
\footnote{This performance comes at the cost of keeping up with the
latest advancements in Desktop technology.}
It has a remarkable number of features for the level of resource
intensity it provides. Being built on the Motif toolkit, it is also one
of the only desktop environments that can integrate well with
high-performance applications over X Forwarding. Neight GTK or Qt are
particularly efficient over the network, making it difficult to
distribute application servers in a client-server fashion. 

CDE’s general development attitude was built not on open-source, but on
open standards technologies. These standards have been around for a
while, and it also means that CDE does a much better job of not getting
in the way, or inneficiently layering over duplicate or redundant
technology for desktop use that is not also part of the core UNIX
system. CDE is a desktop that integrates better with the traditional
UNIX philosphy, while desktop systems like KDE, in particular, have
taken an attitude of developing a number of layering, insulating
technologies that diverge or duplicate existing UNIX functions.
Additionally, KDE tries to hide the underlying mechanisms of the UNIX
operating system, preferring to reimplement features rather than to
expose those features directly to the user. CDE does not do this. 

In summary, CDE is the lightest and most stable full-featured desktop
environment conceivable today.

\heading{Where does CDE go wrong?}

Some of the technologies of CDE requires that the application be aware
of CDE and be coded especially for it. At least in the short term, the
OpenCDE project, and CDE, cannot expect to have this standard position.
Applications are often written with Gnome or KDE in mind, but they will
not, as a whole, likely be written with OpenCDE in mind. Thus, CDE’s
reliance on the applications to provide certain features will not be as
plausible in today’s desktop market as it was when CDE was the standard
desktop.

The style guides of Motif and CDE were only integrated into a single
style guide in the 2.1 versions of both systems. Beyond this, many
applications in the CDE package were not designed for usability as we
now think of it. The workflows on some of the user-level utilities would
not pass today's current standards in usability, though they may have
been considered an improvement back then.

CDE is also out-of-date concerning integration. At the time when it was
receiving the most active development effort CDE needed to worry about
integration and migration from things like XView, OpenWindows, or
OpenLOOK. Today, Gnome, XFCE, GTK, KDE, and Qt represent the standard
applications and environments that CDE needs to worry itself with. 

In other words, there are places and areas where CDE doesn't fulfill its
potential in terms of either consistency or usability. 

\heading{The changing desktop climate}

The climate in which CDE thrived was the commercial, workstation,
enterprise level UNIX labs. These were often high-end labratories doing
things like graphics design or engineering; they might even have been
doing defense and geological research. People using these systems
underwent intensive training, and CDE provided an important reduction
opportunity to those training costs, but the user-base was still largely
technical, even though many did not have degrees in computer science.

In the above climate, preservation of long term investments in
applications (which were not open-source) was of primary importance, as
was ensuring ease of central administration, remote access, and remote
application services. These systems were not concerned with fashion or
the like, though that sometimes did play a part in things.

Today's desktop climate for UNIX machines continues to grow increasingly
complex. Linux still remains a strong server operating system. It is
used throughout, and UNIX operating systems like the BSDs also see heavy
use in the server fields. Workstations are increasingly becoming Windows
or Mac OS machines, and the desktop is seeing an increasing number of
Linux/BSD users. This creates an extremely diverse and competitive
desktop eco-system. There are five major desktop environments that vie
for the attention of users in all sectors ranging from education to
workstations to business to home use. Microsoft Windows and Mac OS X are
the current leaders among many consumer oriented devices. On Linux,
Gnome has a strong holding in many distributions, and has strong support
from enterprise Linux distributions. KDE has a strong following among
consumer and user-level distributions, but OpenSUSE has also made a
strong push for KDE in the enterprise. XFCE is the current lightweight
desktop favorite, built on the GTK toolkit. 

With so many options, it is no wonder that the emphasis has shifted away
from stability in many ways, and on to innovation and rapid progress.
The Open-source development model has created a rapid release cycle with
many users running alpha or beta quality software. Flash and eye candy
are also a significant element in today’s desktop systems. Compiz and
other graphical features make possible very sophisticated graphical
animations as part of the desktop experience. These features also
encourage a great deal of experimentation in how intefaces might
interact. KDE has Workspaces that each have their own set of desktop
widgets, Ubuntu is working in a brand-new unity desktop, and Mac OS X
recently released another version with an emphasis on rapidly viewing
and accessing all elements of a running desktop with a few mouse
gestures. Windows 7 made strong progress forward in terms of how the
user on those systems interact with windows and applications.

In a world of rapid development and progress, stability, ABI
compatibility, and especially preservation of existing investments is
not nearly as important to the average user as it once was. However, it
does remain a strong issue for many sectors, and arguably, this need is
not being met well by existing developments.

Another trend in desktop development is the eagerness to development
integrated systems that are also highly modular. These modular elements
often plug in and out of a aystem, and distributions may or may not
include them. Gnome has a model like this, and KDE, to a lesser degree,
has a model like this. However, Gnome’s extreme use of modularity has
caused some distributions to stop shipping it, since maintaining the
development progress of Gnome became too difficult. 

Many desktop environments also provide or host many backend services far
above and beyond the normal desktop interactions that the user sees,
such as the Akonadi indexing server on KDE. A remarkable number of
backend services, many of which are duplicates of existing OS services
run in conjunction with window managers and file managers.

\heading{What is wrong with the modern desktop?}

\heading{Where does OpenCDE fit in?}

\heading{Workalike versus lookalike}

\heading{The tension between the past and the future}

\heading{Development Goals}

\heading{Development Practices, Norms, and Expectations}

\heading{User expectations and Developer Response}

\heading{Guidelines for deviating from and improving CDE}

\vfill\break

\heading{Bibliography}

\leftskip = 1.3in
\parindent = -0.3in

OpenGroup. {\it Desktop Technologies: CDE}. 
http://www.opengroup.org/cde/. Accessed on 1 August 2011.

\bye
